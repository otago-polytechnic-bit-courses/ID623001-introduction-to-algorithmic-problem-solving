% Author: Grayson Orr
% Course: ID721001: Mobile Application Development

\documentclass{article}
\author{}

\usepackage{graphicx}
\usepackage{wrapfig}
\usepackage{enumerate}
\usepackage{hyperref}
\usepackage[margin = 2.25cm]{geometry}
\usepackage[table]{xcolor}
\usepackage{fancyhdr}
\hypersetup{
  colorlinks = true,
  urlcolor = blue
}
\setlength\parindent{0pt}
\pagestyle{fancy}
\fancyhf{}
\rhead{College of Engineering, Construction \& Living Sciences\\Bachelor of Information Technology}
\lfoot{Portfolio\\Version 1, Semester One, 2022}
\rfoot{\thepage}
 
\begin{document}

\begin{figure}
	\centering
	\includegraphics[width=50mm]{../../resources/img/logo.png}
\end{figure}

\title{College of Engineering, Construction \& Living Sciences\\Bachelor of Information Technology\\ID630151: Introduction to Algorithmic Problem Solving\\Level 6, Credits 15\\\textbf{Portfolio}}
\date{}
\maketitle

\section*{Assessment Overview}
In this \textbf{individual} assessment, you will develop \textbf{four} games using \textbf{C\#} in \textbf{Unity}. You will build the core mechanics for each game in class using the provided \textbf{lecture notes}. You will be given \textbf{assessment tasks} to implement independently. These tasks are at a \textbf{beginner} to \textbf{intermediate level}. In addition, marks will be allocated for code elegance, documentation \& \textbf{Git} usage.

\section*{Learning Outcomes}
At the successful completion of this course, learners will be able to:
\begin{enumerate}
	\item Design \& build usable, attractive games using various introductory algorithms following an appropriate software development methodology.
\end{enumerate}

\section*{Assessment Table}
\renewcommand{\arraystretch}{1.5}
\begin{tabular}{|l|l|l|l|l|}
	\hline
	\vtop{\hbox{\strut \textbf{Assessment}}\hbox{\strut \textbf{Activity}}} & \textbf{Weighting} & \vtop{\hbox{\strut \textbf{Learning}}\hbox{\strut \textbf{Outcomes}}} & \vtop{\hbox{\strut \textbf{Assessment}}\hbox{\strut \textbf{Grading Scheme}}} & \vtop{\hbox{\strut \textbf{Completion}}\hbox{\strut \textbf{Requirements}}} \\

	\hline
	\small Portfolio                                                          & \small 100\%        & \small 1                                                        & \small CRA                                                                    & \small Cumulative                                                           \\ \hline
\end{tabular}

\section*{Conditions of Assessment}
You will complete this assessment during your learner managed time, however, there will be availability during the weekly meetings to discuss the requirements \& your progress of this assessment. This assessment will need to be completed by \textbf{Wednesday, 22 June 2022} at \textbf{5 PM}.

\section*{Pass Criteria}
This assessment is criterion-referenced (CRA) with a cumulative pass mark of \textbf{50\%} over all assessments in \textbf{ID630151: Introduction to Algorithmic Problem Solving}.

\section*{Authenticity}
All parts of your submitted assessment \textbf{must} be completely your work \& any references \textbf{must} be cited appropriately including, externally-sourced graphic elements using \textbf{APA 7th edition}. Provide your references in a \textbf{README.md} file. All media \textbf{must} be royalty free (or legally purchased) for educational use. Failure to do this will result in a mark of \textbf{zero} for this assessment.

\section*{Policy on Submissions, Extensions, Resubmissions \& Resits}
The school's process concerning submissions, extensions, resubmissions \& resits complies with \textbf{Otago Polytechnic} policies. Learners can view policies on the \textbf{Otago Polytechnic} website located at \href{https://www.op.ac.nz/about-us/governance-and-management/policies}{https://www.op.ac.nz/about-us/governance-and-management/policies}.

\section*{Submission}
You \textbf{must} submit all program files via \textbf{GitHub}. The latest program files in the \textbf{master} or \textbf{main} branch will be used to mark against the \textbf{Functionality} criterion. Please test your \textbf{master} or \textbf{main} branch application before you submit. Partial marks \textbf{will not} be given for incomplete functionality. Late submissions will incur a \textbf{10\% penalty per day}, rolling over at \textbf{5:00 PM}.

\section*{Extensions}
Familiarise yourself with the assessment due date. If you need an extension, contact the course lecturer before the due date. If you require more than a week's extension, a medical certificate or support letter from your manager may be needed.

\section*{Resubmissions}
Learners may be requested to resubmit an assessment following a rework of part/s of the original assessment. Resubmissions are to be completed within a negotiable short time frame \& usually \textbf{must} be completed within the timing of the course to which the assessment relates. Resubmissions will be available to learners who have made a genuine attempt at the first assessment opportunity \& achieved a \textbf{D grade (40-49\%)}. The maximum grade awarded for resubmission will be \textbf{C-}.

\section*{Resits}
Resits \& reassessments \textbf{are not} applicable in \textbf{ID630151: Introduction to Algorithmic Problem Solving}.

\section*{Instructions}
You will need to submit games \& documentation that meet the following requirements:

\subsection*{Functionality - Learning Outcomes 1 (70\%)}
\begin{itemize}
	\item Application \textbf{must} open without file structure modification in \textbf{Unity}.
	\item The four games you will create are:
	\begin{itemize}
		\item Introduction to Unity scripting - Sheep Saving (15\%)
		\item Game mechanics - Tower Defence (15\%)
		\item Maze generation - 3D Dungeon Crawler (20\%)
		\item AI strategy - Chess (20\%)
	\end{itemize}
	\item In the \href{https://github.com/otago-polytechnic-bit-courses/ID630151-introduction-to-algorithmic-problem-solving}{course materials repository} on \textbf{GitHub}, you will find the following directories:
	\begin{itemize}
		\item \href{https://github.com/otago-polytechnic-bit-courses/ID630151-introduction-to-algorithmic-problem-solving/tree/main/01-introduction-to-unity-scripting}{01-introduction-to-unity-scripting}
		\item \href{https://github.com/otago-polytechnic-bit-courses/ID630151-introduction-to-algorithmic-problem-solving/tree/main/02-game-mechanics}{02-game-mechanics}
		\item \href{https://github.com/otago-polytechnic-bit-courses/ID630151-introduction-to-algorithmic-problem-solving/tree/main/03-maze-generation}{03-maze-generation}
		\item \href{https://github.com/otago-polytechnic-bit-courses/ID630151-introduction-to-algorithmic-problem-solving/tree/main/04-ai-strategy}{04-ai-strategy}
	\end{itemize}
	\item In each of these directories, you will find additional directories - \textbf{lecture notes}, \textbf{assessment tasks} \& \textbf{advanced assessment tasks}. 
	\begin{itemize}
		\item The \textbf{lecture notes} consist of detailed step-by-step tasks that will help you develop skills \& knowledge in \textbf{Unity} while building a simple game. In addition, you will be introduced to commonly used algorithms in games. \textbf{Note:} If you do not complete these tasks, you will be able to successfully complete the \textbf{assessment tasks}. 
		\item The \textbf{assessment tasks} consist of step-by-step tasks that will help you extend the functionality of your game. However, these tasks are not as detailed as the \textbf{lecture notes}.
		\item The \textbf{advanced assessment tasks} consist of \textbf{independent research} \& \textbf{problem-solving} tasks that will help you extend the functionality of your game to an \textbf{intermediate level}. You will complete these tasks in your own learner managed time.
	\end{itemize} 
\end{itemize}

\subsection*{Code Elegance - Learning Outcomes 1 (20\%)}
\begin{itemize}
	\item Use of intermediate variables, i.e., no function calls as arguments.
	\item Idiomatic use of control flow, data structures \& in-built functions.
	\item Efficient algorithmic approach.
	\item Sufficient modularity. 
	\item File header comments. You \textbf{need} to explain the purpose of each \textbf{script} file.
	\item In-line comments. You \textbf{need} to explain complex logic in each \textbf{script} file that is not obvious.
	\item \textbf{Script} files are formatted.
	\item No dead or unused code.
\end{itemize}

\subsection*{Documentation \& Git/GitHub Usage - Learning Outcomes 1 (10\%)}
\begin{itemize}
	\item Provide the following in your repository \textbf{README.md} file:
	      \begin{itemize}
		      \item URL to your games online.
		      \item URLs to resources used to build your game, i.e., \textbf{StackOverflow} posts, \textbf{Unity Forum} posts, etc.
	      \end{itemize}
	\item Commit messages \textbf{must} reflect the context of each functional requirement change.
\end{itemize}

\subsection*{Additional Information}
\begin{itemize}
	\item Attempt to commit at least \textbf{10} times per week.
	\item \textbf{Do not} rewrite your \textbf{Git} history. It is important that the course lecturer can see how you worked on your assessment over time.
\end{itemize}

\end{document}