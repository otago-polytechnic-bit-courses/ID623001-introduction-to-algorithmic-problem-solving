% Author: Grayson Orr
% Course: ID623001: Introduction to Algorithmic Problem Solving

\documentclass{article}
\author{}

\usepackage{fontspec}
\setmainfont{Arial}

\usepackage{graphicx}
\usepackage{wrapfig}
\usepackage{enumerate}
\usepackage{hyperref}
\usepackage[margin = 2.25cm]{geometry}
\usepackage[table]{xcolor}
\usepackage{fancyhdr}
\hypersetup{
  colorlinks = true,
  urlcolor = blue
}
\setlength\parindent{0pt}
\pagestyle{fancy}
\fancyhf{}
\rhead{College of Engineering, Construction and Living Sciences\\Bachelor of Information Technology}
\lfoot{Portfolio\\Version 3, Semester One, 2024}
\rfoot{\thepage}
 
\begin{document}

\begin{figure}
	\centering
	\includegraphics[width=50mm]{../../resources/img/logo.png}
\end{figure}

\title{College of Engineering, Construction and Living Sciences\\Bachelor of Information Technology\\ID623001: Introduction to Algorithmic Problem Solving\\Level 6, Credits 15\\\textbf{Portfolio}}
\date{}
\maketitle

\section*{Assessment Overview}
In this \textbf{individual} assessment, you will develop \textbf{four} games using \textbf{C\#} and \textbf{Unity}. You will build the core mechanics for each game in class using the provided \textbf{lecture notes}. You will be given \textbf{assessment tasks} to implement in your learner-managed time. These tasks are at a beginner to intermediate level. In addition, marks will be allocated for code quality and best practices, and documentation.

\section*{Learning Outcomes}
At the successful completion of this course, learners will be able to:
\begin{enumerate}
	\item Design and build usable, attractive games using various introductory algorithms following an appropriate software development methodology.
\end{enumerate}

\section*{Assessments}
\renewcommand{\arraystretch}{1.5}
\begin{tabular}{|c|c|c|c|}
	\hline
	\textbf{Assessment}                                 & \textbf{Weighting} & \textbf{Due Date}            & \textbf{Learning Outcomes} \\ \hline
	\small Portfolio                        & \small 100\%        & \small 21-06-2024 (Friday at 4.59 PM)  & \small 1                   \\ \hline
\end{tabular}

\section*{Conditions of Assessment}
You will complete this assessment during your learner-managed time. However, there will be time during class to discuss the requirements and your progress on this assessment. This assessment will need to be completed by \textbf{Friday, 21 June 2024} at \textbf{4.59 PM}. 

\section*{Pass Criteria}
This assessment is criterion-referenced (CRA) with a cumulative pass mark of \textbf{50\%} over all assessments in \textbf{ID623001: Introduction to Algorithmic Problem Solving}.

\section*{Authenticity}
All parts of your submitted assessment \textbf{must} be completely your work and any references \textbf{must} be cited appropriately including, externally-sourced graphic elements using \textbf{APA 7th edition}. Provide your references in a \textbf{README.md} file. All media \textbf{must} be royalty free (or legally purchased) for educational use. Failure to do this will result in a mark of \textbf{zero} for this assessment.

\section*{Policy on Submissions, Extensions, Resubmissions and Resits}
The school's process concerning submissions, extensions, resubmissions and resits complies with \textbf{Otago Polytechnic} policies. Learners can view policies on the \textbf{Otago Polytechnic} website located at \href{https://www.op.ac.nz/about-us/governance-and-management/policies}{https://www.op.ac.nz/about-us/governance-and-management/policies}.

\section*{Submission}
You \textbf{must} submit all application files via \textbf{GitHub Classroom}. Here is the URL to the repository you will use for your submission – \href{https://classroom.github.com/a/-8gjb\_cN}{https://classroom.github.com/a/-8gjb\_cN}. If you do not have not one, create a \textbf{.gitignore} and add the ignored files in this resource - \href{https://raw.githubusercontent.com/github/gitignore/main/Unity.gitignore}{https://raw.githubusercontent.com/github/gitignore/main/Unity.gitignore}. The latest application files in the \textbf{main} branch will be used to mark against the \textbf{Functionality} criterion. Please test before you submit. Partial marks \textbf{will not} be given for incomplete functionality. Late submissions will incur a \textbf{10\% penalty per day}, rolling over at \textbf{5:00 PM}. 

\section*{Extensions}
Familiarise yourself with the assessment due date. Extensions will \textbf{only} be granted if you are unable to complete the assessment by the due date because of \textbf{unforeseen circumstances outside your control}. The length of the extension granted will depend on the circumstances and \textbf{must} be negotiated with the course lecturer before the assessment due date. A medical certificate or support letter may be needed. Extensions will not be granted for poor time management or pressure of other assessments.

\section*{Resits}
Resits and reassessments \textbf{are not} applicable in \textbf{ID623001: Introduction to Algorithmic Problem Solving}.

\section*{Instructions}
\subsection*{Functionality - Learning Outcome 1 (50\%)}
\begin{itemize}
	\item Application \textbf{must} open without code or file structure modification in \textbf{Unity}.
	\item The four games you will create are:
	\begin{itemize}
		\item Introduction to Unity scripting - Sheep Saving (25\%)
		\item Game mechanics - Tower Defence (25\%)
		\item Maze generation - 3D Dungeon Crawler (25\%)
		\item AI strategy - Chess (25\%)
	\end{itemize}
	\item In the \href{https://github.com/otago-polytechnic-bit-courses/ID623001-introduction-to-algorithmic-problem-solving}{course materials repository} on \textbf{GitHub}, you will find the following directories:
	\begin{itemize}
		\item \href{https://github.com/otago-polytechnic-bit-courses/ID623001-introduction-to-algorithmic-problem-solving/tree/main/01-introduction-to-unity-scripting}{01-introduction-to-unity-scripting}
		\item \href{https://github.com/otago-polytechnic-bit-courses/ID623001-introduction-to-algorithmic-problem-solving/tree/main/02-game-mechanics}{02-game-mechanics}
		\item \href{https://github.com/otago-polytechnic-bit-courses/ID623001-introduction-to-algorithmic-problem-solving/tree/main/03-maze-generation}{03-maze-generation}
		\item \href{https://github.com/otago-polytechnic-bit-courses/ID623001-introduction-to-algorithmic-problem-solving/tree/main/04-ai-strategy}{04-ai-strategy}
	\end{itemize}
	\item In each of these directories, you will find additional directories - \textbf{lecture notes} and \textbf{assessment tasks}. 
	\begin{itemize}
		\item The \textbf{lecture notes} consist of detailed step-by-step tasks that will help you develop skills and knowledge in \textbf{Unity} while building a simple game. In addition, you will be introduced to commonly used algorithms in games.
		\item The \textbf{assessment tasks} consist of step-by-step tasks that will help you extend the functionality of your game. However, these tasks are not as detailed as the \textbf{lecture notes}. The \textbf{advanced assessment tasks} consist of \textbf{independent research} tasks that will help you extend the functionality of your game to an \textbf{intermediate level}.
	\end{itemize} 
\end{itemize}

\subsection*{Code Quality and Best Practices - Learning Outcome 1 (45\%)}
\begin{itemize}
    \item A \textbf{Unity} \textbf{.gitignore} file is used. 
    \item Appropriate naming of files, variables, methods and classes.
    \item Idiomatic use of values, control flow, data structures and in-built functions.
    \item Efficient algorithmic approach.
    \item Sufficient modularity.
    \item Each file has an \textbf{XML documentation comment} located at the top of the file. In the \textbf{root} directory of the \textbf{course materials} repository, you will find an \textbf{XML documentation comment} example in the \textbf{xml-documentation-comment.txt} file.
    \item Formatted code.
    \item No dead or unused code.
\end{itemize} 

\subsection*{Documentation - Learning Outcome 1 (5\%)}
\begin{itemize}
	\item A \textbf{GitHub} project board to help you organise and prioritise your development work. The course lecturer needs to see consistent use of the \textbf{GitHub} project board for the duration of the assessment.
\end{itemize} 

\subsection*{Additional Information}
\begin{itemize}
    \item \textbf{Do not} rewrite your \textbf{Git} history. It is important that the course lecturer can see how you worked on your assessment over time. 
    \item You need to show the course lecturer the initial \textbf{GitHub} project board before you start your development work. Following this, you need to show the course lecturer your \textbf{GitHub} project board at the end of each week.
\end{itemize} 
\end{document}